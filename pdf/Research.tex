
\documentclass[9pt, margin]{res}
% Use the res.cls style, the font size can be changed to 11pt or 12pt here
\usepackage{hyperref}
\usepackage{setspace}
\hypersetup{colorlinks=true, linkcolor=blue, filecolor=magenta, urlcolor=cyan}
\usepackage{endnotes}

\let\footnote=\endnote
\usepackage{helvet} % Default font is the helvetica postscript font
%\usepackage{newcent} % To change the default font to the new century schoolbook postscript font uncomment this line and comment the one above
%\topmargin=0in
\oddsidemargin -1in
%\evensidemargin -.5in
\setlength{\textwidth}{6.3in} % Text width of the document
%\setlength{\textheight}{650 pt}


\begin{document}
\moveleft.5\hoffset\centerline{\large\bf Research Summary} % Your name at the top


\begin{resume}

{\sl Chemical Physics Theory Group, University of Toronto}  \hfill Fall 2017 -- Spring 2018 \\
Senior thesis project under the supervision of \href{http://www.chem.utoronto.ca/~jmschofi/}{Professor Jeremy Schofield}

\textit{Topic:} Using Fokker-Plank dynamics to model protein folding\\
\textit{Work done so far:} Validated the model using exact solution of the Focker-Planck equation for the simple potentials. Moving towards studying sharp, but continuous potentials.\\
\newline \newline \newline
{\sl Chemical Physics Theory Group, University of Toronto} \hfill Summer 2016 -- Fall 2017 \\
Two summer projects, funded by Excellence Research Fund (both) and Center for Quantum Information and Quantum Control (the second one) under the supervision of \href{http://www.chem.utoronto.ca/~dsegal/index.html}{Professor Dvira Segal}

\textit{Work done:}

\begin{enumerate}
\item Going beyond Landauer (scattering) formalism with the help of Butticker probes to describe charge transport in the intermediate quantum-classical regime. System studied: single molecule DNA junctions under thermoelectric bias.
\item Continuing investigation of the junctions based on the DNA and other polymers; preparation of our quantum transport code for the publication; spin filtering using the chirality of the DNA helix; pushing the limits of minimum models for quantum refrigerator: investigation of a two-level system with three reservoirs under the strong system-bath coupling. 
\end{enumerate}

\textit{Publications and Presentations:}
\begin{itemize}
	\item \textbf{Korol R.}; Segal D. Machine Learning Prediction of DNA Charge Transport. \textit{J. Phys. Chem. B}, \textbf{2019}, 123 (13), pp 2801 –- 2811. \href{https://pubs.acs.org/doi/full/10.1021/acs.jpcb.8b12557}{10.1021/acs.jpcb.8b12557}
		\begin{itemize}
			\item Poster at the Berkeley Mini Stat Mech \href{http://berkeleystatmech.org/}{Meeting} (2018)
		\end{itemize}
	\item \textbf{Korol, R.}; Kilgour,  M.; Segal, D. Thermopower Of Molecular Junctions: Tunneling To Hopping Crossover In DNA. \textit{J. Chem. Phys} 145, 224702 \textbf{2016}. \href{http://aip.scitation.org/doi/10.1063/1.4971167}{10.1063/1.4971167}
		\begin{itemize}
			\item Contributed talk at 45\textsuperscript{th} Southern Ontario Undergraduate Student Chemistry \href{http://www.souscc45.ca}{ Conference} (York University)
			\item Contributed talk at Chemical Biophysics \href{www.chembiophys.ca}{Symposium}--2017 (University of Toronto)
			\item Poster at the 100\textsuperscript{th} Canadian Chemistry \href{www.csc2017.ca}{Conference} (Toronto)
		\end{itemize} 
	\item \textbf{Korol, R.}; Kilgour,  M.; Segal, D. ProbeZT: Simulation of transport coefficients of molecular electronic junctions under environmental effects using Buttiker’s probes. \textit{Comp. Phys. Comm.} (in press) \textbf{2017} \href{https://doi.org/10.1016/j.cpc.2017.10.005}{10.1016/j.cpc.2017.10.005}
	\item \textbf{Korol R.}; Segal D. Electrical conduction through DNA molecules: An exhaustive computational study (manuscript in preparation)
	\begin{itemize}
		\item Contributed talk at \href{http://scp.uwaterloo.ca}{33\textsuperscript{rd} Symposium} on Chemical Physics, U of Waterloo
		\newline \newline
	\end{itemize}
\end{itemize}
{\sl Inorganic Synthetic Laboratory, University of Toronto  \hfill Winter -- Spring 2016 \\
Volunteering under direct supervision of Dr. Lauren Longobardi, PI: \href{http://www.chem.utoronto.ca/staff/DSTEPHAN/index.htm}{Professor Doug Stephan}} 

\textit{Work done:}
Synthesis of radicals containing Boron (in the glove due to water and air-sensitivity of the reagents and products), running the reaction scopes, NMR analysis, various separations.

\textit{Publication:}
\begin{itemize}
	\item Longobardi, L.E.; Zatsepin, P.; \textbf{Korol, R.}; Liu, L.; Grimme, S.; Stephan D.W. Reactions Of Boron-Derived Radicals With Nucleophiles. \textit{J. Am. Chem. Soc.} \textbf{2016} 139 (1), pp 426–-435. \href{http://pubs.acs.org/doi/abs/10.1021%2Fjacs.6b11190}{10.1021/jacs.6b11190} 
\end{itemize}

{\sl Organic Materials Laboratory, Weizmann Institute of Science, Rehovot, Israel \hfill Summer 2015 \\
\href{https://www.weizmann.ac.il/feinberg/admissions/kupcinet-getz-international-Summer-school/about-program-0}{Kupcinet-Getz Summer School} under the supervision of \href{https://www.weizmann.ac.il/Organic_Chemistry/Rybtchinski/}{Professor Boris Rybtchinski}}\\
\textit{Topic:} self-assembly of organic nanocrystals, their fluorescence and non-linear optics

\textit{Work done:} Synthesis and purification of the perylene diimide (PDI) dye with a non-centrosymmetric lattice, UV-VIS and fluorescence spectroscopy studying its optical properties in various solvents; SEM and TEM imaging of the self-assemblies (under the supervision of Shacked Rosenne and Dr. Haim Weissman) 
\newline \newline

{\sl Inorganic Synthetic Laboratory, University of Toronto Mississauga \hfill Fall 2014 -- Spring 2015 \\
Volunteering under the supervision of David Armstrong, PI: \href{http://www.utm.utoronto.ca/fekl/}{Associate Professor Ulrich Fekl}}\\
\textit{Topic:} Functionalization of halogenated adamantanes

\textit{Work done:} Synthesis of mono- and dibromoadamantane, reacting these with the alkyl metal nucleophiles.  
\newline\newline\newline

{\sl Inorganic materials Laboratory, Eastern European National University, Lutsk, Ukraine} 
\newline Summer 2013 -- Spring 2014 \\
A project in the Junior Academy of Sciences of Ukraine under the supervision of \href{https://scholar.google.com.ua/citations?user=m7yvotUAAAAJ&hl=ru}{Dr. Oleksandr Yanchuk}

\textit{Work done:} Synthesis of nanoparticles of ZnO using a two-electrode electrolytic cell set-up under various conditions.  
\textit{Publications and presentations:}
\begin{itemize}
	\item \textbf{Korol, R.}; Marchuk, V.; Urubkov I.V.; Yanchuk O.M. Controlling the Size and Morphology of ZnO Nanorods in Two-electrode Synthesis Using Auxiliary Stabilizers
	(manuscript in preparation)
	\begin{itemize}
		\item Poster at the National Ecology Olympiad, Vinnytsa, Ukraine
		\item Poster at the Intel-Eco Ukraine - the national stage of the international Intel ISEF, Kiev, Ukraine
		\item Contributed talk at the National competition, organized by the Junior Academy of Sciences of Ukraine
		\newline\newline
	\end{itemize}
\end{itemize}
{\sl Biological Chemistry Laboratory, Eastern European National University, Lutsk, Ukraine} 
\newline \hfill Summer 2012 -- Spring 2013 \\
A project in the Junior Academy of Sciences of Ukraine under the supervision of Dr. Vasyl Voytiuk and Dr. Halyna Yagenska

\textit{Topic:} Plant Leaves Morphology and Biochemistry in the Urban Atmosphere

\textit{Work done}: sample gathering and preparation (fieldwork), extraction of pigments, UV-VIS spectroscopic analysis (labwork)
  
\textit{Publications and presentations:}
\begin{itemize}
	\item \textbf{Korol, R.}; Repetylo, I.; Yagenska H. Plant Leaves Morphology and Biochemistry in the Urban Atmosphere. 21\textsuperscript{st} International Environmental Project Olympiad project book, 2013, Istanbul, Turkey
	\begin{itemize}
		\item Poster at the INEPO-2013, Instanbul, Turkey
	\end{itemize}
\end{itemize}

\end{resume}
\end{document}